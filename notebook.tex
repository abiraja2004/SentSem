
% Default to the notebook output style

    


% Inherit from the specified cell style.




    
\documentclass[11pt]{article}

    
    
    \usepackage[T1]{fontenc}
    % Nicer default font (+ math font) than Computer Modern for most use cases
    \usepackage{mathpazo}

    % Basic figure setup, for now with no caption control since it's done
    % automatically by Pandoc (which extracts ![](path) syntax from Markdown).
    \usepackage{graphicx}
    % We will generate all images so they have a width \maxwidth. This means
    % that they will get their normal width if they fit onto the page, but
    % are scaled down if they would overflow the margins.
    \makeatletter
    \def\maxwidth{\ifdim\Gin@nat@width>\linewidth\linewidth
    \else\Gin@nat@width\fi}
    \makeatother
    \let\Oldincludegraphics\includegraphics
    % Set max figure width to be 80% of text width, for now hardcoded.
    \renewcommand{\includegraphics}[1]{\Oldincludegraphics[width=.8\maxwidth]{#1}}
    % Ensure that by default, figures have no caption (until we provide a
    % proper Figure object with a Caption API and a way to capture that
    % in the conversion process - todo).
    \usepackage{caption}
    \DeclareCaptionLabelFormat{nolabel}{}
    \captionsetup{labelformat=nolabel}

    \usepackage{adjustbox} % Used to constrain images to a maximum size 
    \usepackage{xcolor} % Allow colors to be defined
    \usepackage{enumerate} % Needed for markdown enumerations to work
    \usepackage{geometry} % Used to adjust the document margins
    \usepackage{amsmath} % Equations
    \usepackage{amssymb} % Equations
    \usepackage{textcomp} % defines textquotesingle
    % Hack from http://tex.stackexchange.com/a/47451/13684:
    \AtBeginDocument{%
        \def\PYZsq{\textquotesingle}% Upright quotes in Pygmentized code
    }
    \usepackage{upquote} % Upright quotes for verbatim code
    \usepackage{eurosym} % defines \euro
    \usepackage[mathletters]{ucs} % Extended unicode (utf-8) support
    \usepackage[utf8x]{inputenc} % Allow utf-8 characters in the tex document
    \usepackage{fancyvrb} % verbatim replacement that allows latex
    \usepackage{grffile} % extends the file name processing of package graphics 
                         % to support a larger range 
    % The hyperref package gives us a pdf with properly built
    % internal navigation ('pdf bookmarks' for the table of contents,
    % internal cross-reference links, web links for URLs, etc.)
    \usepackage{hyperref}
    \usepackage{longtable} % longtable support required by pandoc >1.10
    \usepackage{booktabs}  % table support for pandoc > 1.12.2
    \usepackage[inline]{enumitem} % IRkernel/repr support (it uses the enumerate* environment)
    \usepackage[normalem]{ulem} % ulem is needed to support strikethroughs (\sout)
                                % normalem makes italics be italics, not underlines
    

    
    
    % Colors for the hyperref package
    \definecolor{urlcolor}{rgb}{0,.145,.698}
    \definecolor{linkcolor}{rgb}{.71,0.21,0.01}
    \definecolor{citecolor}{rgb}{.12,.54,.11}

    % ANSI colors
    \definecolor{ansi-black}{HTML}{3E424D}
    \definecolor{ansi-black-intense}{HTML}{282C36}
    \definecolor{ansi-red}{HTML}{E75C58}
    \definecolor{ansi-red-intense}{HTML}{B22B31}
    \definecolor{ansi-green}{HTML}{00A250}
    \definecolor{ansi-green-intense}{HTML}{007427}
    \definecolor{ansi-yellow}{HTML}{DDB62B}
    \definecolor{ansi-yellow-intense}{HTML}{B27D12}
    \definecolor{ansi-blue}{HTML}{208FFB}
    \definecolor{ansi-blue-intense}{HTML}{0065CA}
    \definecolor{ansi-magenta}{HTML}{D160C4}
    \definecolor{ansi-magenta-intense}{HTML}{A03196}
    \definecolor{ansi-cyan}{HTML}{60C6C8}
    \definecolor{ansi-cyan-intense}{HTML}{258F8F}
    \definecolor{ansi-white}{HTML}{C5C1B4}
    \definecolor{ansi-white-intense}{HTML}{A1A6B2}

    % commands and environments needed by pandoc snippets
    % extracted from the output of `pandoc -s`
    \providecommand{\tightlist}{%
      \setlength{\itemsep}{0pt}\setlength{\parskip}{0pt}}
    \DefineVerbatimEnvironment{Highlighting}{Verbatim}{commandchars=\\\{\}}
    % Add ',fontsize=\small' for more characters per line
    \newenvironment{Shaded}{}{}
    \newcommand{\KeywordTok}[1]{\textcolor[rgb]{0.00,0.44,0.13}{\textbf{{#1}}}}
    \newcommand{\DataTypeTok}[1]{\textcolor[rgb]{0.56,0.13,0.00}{{#1}}}
    \newcommand{\DecValTok}[1]{\textcolor[rgb]{0.25,0.63,0.44}{{#1}}}
    \newcommand{\BaseNTok}[1]{\textcolor[rgb]{0.25,0.63,0.44}{{#1}}}
    \newcommand{\FloatTok}[1]{\textcolor[rgb]{0.25,0.63,0.44}{{#1}}}
    \newcommand{\CharTok}[1]{\textcolor[rgb]{0.25,0.44,0.63}{{#1}}}
    \newcommand{\StringTok}[1]{\textcolor[rgb]{0.25,0.44,0.63}{{#1}}}
    \newcommand{\CommentTok}[1]{\textcolor[rgb]{0.38,0.63,0.69}{\textit{{#1}}}}
    \newcommand{\OtherTok}[1]{\textcolor[rgb]{0.00,0.44,0.13}{{#1}}}
    \newcommand{\AlertTok}[1]{\textcolor[rgb]{1.00,0.00,0.00}{\textbf{{#1}}}}
    \newcommand{\FunctionTok}[1]{\textcolor[rgb]{0.02,0.16,0.49}{{#1}}}
    \newcommand{\RegionMarkerTok}[1]{{#1}}
    \newcommand{\ErrorTok}[1]{\textcolor[rgb]{1.00,0.00,0.00}{\textbf{{#1}}}}
    \newcommand{\NormalTok}[1]{{#1}}
    
    % Additional commands for more recent versions of Pandoc
    \newcommand{\ConstantTok}[1]{\textcolor[rgb]{0.53,0.00,0.00}{{#1}}}
    \newcommand{\SpecialCharTok}[1]{\textcolor[rgb]{0.25,0.44,0.63}{{#1}}}
    \newcommand{\VerbatimStringTok}[1]{\textcolor[rgb]{0.25,0.44,0.63}{{#1}}}
    \newcommand{\SpecialStringTok}[1]{\textcolor[rgb]{0.73,0.40,0.53}{{#1}}}
    \newcommand{\ImportTok}[1]{{#1}}
    \newcommand{\DocumentationTok}[1]{\textcolor[rgb]{0.73,0.13,0.13}{\textit{{#1}}}}
    \newcommand{\AnnotationTok}[1]{\textcolor[rgb]{0.38,0.63,0.69}{\textbf{\textit{{#1}}}}}
    \newcommand{\CommentVarTok}[1]{\textcolor[rgb]{0.38,0.63,0.69}{\textbf{\textit{{#1}}}}}
    \newcommand{\VariableTok}[1]{\textcolor[rgb]{0.10,0.09,0.49}{{#1}}}
    \newcommand{\ControlFlowTok}[1]{\textcolor[rgb]{0.00,0.44,0.13}{\textbf{{#1}}}}
    \newcommand{\OperatorTok}[1]{\textcolor[rgb]{0.40,0.40,0.40}{{#1}}}
    \newcommand{\BuiltInTok}[1]{{#1}}
    \newcommand{\ExtensionTok}[1]{{#1}}
    \newcommand{\PreprocessorTok}[1]{\textcolor[rgb]{0.74,0.48,0.00}{{#1}}}
    \newcommand{\AttributeTok}[1]{\textcolor[rgb]{0.49,0.56,0.16}{{#1}}}
    \newcommand{\InformationTok}[1]{\textcolor[rgb]{0.38,0.63,0.69}{\textbf{\textit{{#1}}}}}
    \newcommand{\WarningTok}[1]{\textcolor[rgb]{0.38,0.63,0.69}{\textbf{\textit{{#1}}}}}
    
    
    % Define a nice break command that doesn't care if a line doesn't already
    % exist.
    \def\br{\hspace*{\fill} \\* }
    % Math Jax compatability definitions
    \def\gt{>}
    \def\lt{<}
    % Document parameters
    \title{SentSem}
    
    
    

    % Pygments definitions
    
\makeatletter
\def\PY@reset{\let\PY@it=\relax \let\PY@bf=\relax%
    \let\PY@ul=\relax \let\PY@tc=\relax%
    \let\PY@bc=\relax \let\PY@ff=\relax}
\def\PY@tok#1{\csname PY@tok@#1\endcsname}
\def\PY@toks#1+{\ifx\relax#1\empty\else%
    \PY@tok{#1}\expandafter\PY@toks\fi}
\def\PY@do#1{\PY@bc{\PY@tc{\PY@ul{%
    \PY@it{\PY@bf{\PY@ff{#1}}}}}}}
\def\PY#1#2{\PY@reset\PY@toks#1+\relax+\PY@do{#2}}

\expandafter\def\csname PY@tok@w\endcsname{\def\PY@tc##1{\textcolor[rgb]{0.73,0.73,0.73}{##1}}}
\expandafter\def\csname PY@tok@c\endcsname{\let\PY@it=\textit\def\PY@tc##1{\textcolor[rgb]{0.25,0.50,0.50}{##1}}}
\expandafter\def\csname PY@tok@cp\endcsname{\def\PY@tc##1{\textcolor[rgb]{0.74,0.48,0.00}{##1}}}
\expandafter\def\csname PY@tok@k\endcsname{\let\PY@bf=\textbf\def\PY@tc##1{\textcolor[rgb]{0.00,0.50,0.00}{##1}}}
\expandafter\def\csname PY@tok@kp\endcsname{\def\PY@tc##1{\textcolor[rgb]{0.00,0.50,0.00}{##1}}}
\expandafter\def\csname PY@tok@kt\endcsname{\def\PY@tc##1{\textcolor[rgb]{0.69,0.00,0.25}{##1}}}
\expandafter\def\csname PY@tok@o\endcsname{\def\PY@tc##1{\textcolor[rgb]{0.40,0.40,0.40}{##1}}}
\expandafter\def\csname PY@tok@ow\endcsname{\let\PY@bf=\textbf\def\PY@tc##1{\textcolor[rgb]{0.67,0.13,1.00}{##1}}}
\expandafter\def\csname PY@tok@nb\endcsname{\def\PY@tc##1{\textcolor[rgb]{0.00,0.50,0.00}{##1}}}
\expandafter\def\csname PY@tok@nf\endcsname{\def\PY@tc##1{\textcolor[rgb]{0.00,0.00,1.00}{##1}}}
\expandafter\def\csname PY@tok@nc\endcsname{\let\PY@bf=\textbf\def\PY@tc##1{\textcolor[rgb]{0.00,0.00,1.00}{##1}}}
\expandafter\def\csname PY@tok@nn\endcsname{\let\PY@bf=\textbf\def\PY@tc##1{\textcolor[rgb]{0.00,0.00,1.00}{##1}}}
\expandafter\def\csname PY@tok@ne\endcsname{\let\PY@bf=\textbf\def\PY@tc##1{\textcolor[rgb]{0.82,0.25,0.23}{##1}}}
\expandafter\def\csname PY@tok@nv\endcsname{\def\PY@tc##1{\textcolor[rgb]{0.10,0.09,0.49}{##1}}}
\expandafter\def\csname PY@tok@no\endcsname{\def\PY@tc##1{\textcolor[rgb]{0.53,0.00,0.00}{##1}}}
\expandafter\def\csname PY@tok@nl\endcsname{\def\PY@tc##1{\textcolor[rgb]{0.63,0.63,0.00}{##1}}}
\expandafter\def\csname PY@tok@ni\endcsname{\let\PY@bf=\textbf\def\PY@tc##1{\textcolor[rgb]{0.60,0.60,0.60}{##1}}}
\expandafter\def\csname PY@tok@na\endcsname{\def\PY@tc##1{\textcolor[rgb]{0.49,0.56,0.16}{##1}}}
\expandafter\def\csname PY@tok@nt\endcsname{\let\PY@bf=\textbf\def\PY@tc##1{\textcolor[rgb]{0.00,0.50,0.00}{##1}}}
\expandafter\def\csname PY@tok@nd\endcsname{\def\PY@tc##1{\textcolor[rgb]{0.67,0.13,1.00}{##1}}}
\expandafter\def\csname PY@tok@s\endcsname{\def\PY@tc##1{\textcolor[rgb]{0.73,0.13,0.13}{##1}}}
\expandafter\def\csname PY@tok@sd\endcsname{\let\PY@it=\textit\def\PY@tc##1{\textcolor[rgb]{0.73,0.13,0.13}{##1}}}
\expandafter\def\csname PY@tok@si\endcsname{\let\PY@bf=\textbf\def\PY@tc##1{\textcolor[rgb]{0.73,0.40,0.53}{##1}}}
\expandafter\def\csname PY@tok@se\endcsname{\let\PY@bf=\textbf\def\PY@tc##1{\textcolor[rgb]{0.73,0.40,0.13}{##1}}}
\expandafter\def\csname PY@tok@sr\endcsname{\def\PY@tc##1{\textcolor[rgb]{0.73,0.40,0.53}{##1}}}
\expandafter\def\csname PY@tok@ss\endcsname{\def\PY@tc##1{\textcolor[rgb]{0.10,0.09,0.49}{##1}}}
\expandafter\def\csname PY@tok@sx\endcsname{\def\PY@tc##1{\textcolor[rgb]{0.00,0.50,0.00}{##1}}}
\expandafter\def\csname PY@tok@m\endcsname{\def\PY@tc##1{\textcolor[rgb]{0.40,0.40,0.40}{##1}}}
\expandafter\def\csname PY@tok@gh\endcsname{\let\PY@bf=\textbf\def\PY@tc##1{\textcolor[rgb]{0.00,0.00,0.50}{##1}}}
\expandafter\def\csname PY@tok@gu\endcsname{\let\PY@bf=\textbf\def\PY@tc##1{\textcolor[rgb]{0.50,0.00,0.50}{##1}}}
\expandafter\def\csname PY@tok@gd\endcsname{\def\PY@tc##1{\textcolor[rgb]{0.63,0.00,0.00}{##1}}}
\expandafter\def\csname PY@tok@gi\endcsname{\def\PY@tc##1{\textcolor[rgb]{0.00,0.63,0.00}{##1}}}
\expandafter\def\csname PY@tok@gr\endcsname{\def\PY@tc##1{\textcolor[rgb]{1.00,0.00,0.00}{##1}}}
\expandafter\def\csname PY@tok@ge\endcsname{\let\PY@it=\textit}
\expandafter\def\csname PY@tok@gs\endcsname{\let\PY@bf=\textbf}
\expandafter\def\csname PY@tok@gp\endcsname{\let\PY@bf=\textbf\def\PY@tc##1{\textcolor[rgb]{0.00,0.00,0.50}{##1}}}
\expandafter\def\csname PY@tok@go\endcsname{\def\PY@tc##1{\textcolor[rgb]{0.53,0.53,0.53}{##1}}}
\expandafter\def\csname PY@tok@gt\endcsname{\def\PY@tc##1{\textcolor[rgb]{0.00,0.27,0.87}{##1}}}
\expandafter\def\csname PY@tok@err\endcsname{\def\PY@bc##1{\setlength{\fboxsep}{0pt}\fcolorbox[rgb]{1.00,0.00,0.00}{1,1,1}{\strut ##1}}}
\expandafter\def\csname PY@tok@kc\endcsname{\let\PY@bf=\textbf\def\PY@tc##1{\textcolor[rgb]{0.00,0.50,0.00}{##1}}}
\expandafter\def\csname PY@tok@kd\endcsname{\let\PY@bf=\textbf\def\PY@tc##1{\textcolor[rgb]{0.00,0.50,0.00}{##1}}}
\expandafter\def\csname PY@tok@kn\endcsname{\let\PY@bf=\textbf\def\PY@tc##1{\textcolor[rgb]{0.00,0.50,0.00}{##1}}}
\expandafter\def\csname PY@tok@kr\endcsname{\let\PY@bf=\textbf\def\PY@tc##1{\textcolor[rgb]{0.00,0.50,0.00}{##1}}}
\expandafter\def\csname PY@tok@bp\endcsname{\def\PY@tc##1{\textcolor[rgb]{0.00,0.50,0.00}{##1}}}
\expandafter\def\csname PY@tok@fm\endcsname{\def\PY@tc##1{\textcolor[rgb]{0.00,0.00,1.00}{##1}}}
\expandafter\def\csname PY@tok@vc\endcsname{\def\PY@tc##1{\textcolor[rgb]{0.10,0.09,0.49}{##1}}}
\expandafter\def\csname PY@tok@vg\endcsname{\def\PY@tc##1{\textcolor[rgb]{0.10,0.09,0.49}{##1}}}
\expandafter\def\csname PY@tok@vi\endcsname{\def\PY@tc##1{\textcolor[rgb]{0.10,0.09,0.49}{##1}}}
\expandafter\def\csname PY@tok@vm\endcsname{\def\PY@tc##1{\textcolor[rgb]{0.10,0.09,0.49}{##1}}}
\expandafter\def\csname PY@tok@sa\endcsname{\def\PY@tc##1{\textcolor[rgb]{0.73,0.13,0.13}{##1}}}
\expandafter\def\csname PY@tok@sb\endcsname{\def\PY@tc##1{\textcolor[rgb]{0.73,0.13,0.13}{##1}}}
\expandafter\def\csname PY@tok@sc\endcsname{\def\PY@tc##1{\textcolor[rgb]{0.73,0.13,0.13}{##1}}}
\expandafter\def\csname PY@tok@dl\endcsname{\def\PY@tc##1{\textcolor[rgb]{0.73,0.13,0.13}{##1}}}
\expandafter\def\csname PY@tok@s2\endcsname{\def\PY@tc##1{\textcolor[rgb]{0.73,0.13,0.13}{##1}}}
\expandafter\def\csname PY@tok@sh\endcsname{\def\PY@tc##1{\textcolor[rgb]{0.73,0.13,0.13}{##1}}}
\expandafter\def\csname PY@tok@s1\endcsname{\def\PY@tc##1{\textcolor[rgb]{0.73,0.13,0.13}{##1}}}
\expandafter\def\csname PY@tok@mb\endcsname{\def\PY@tc##1{\textcolor[rgb]{0.40,0.40,0.40}{##1}}}
\expandafter\def\csname PY@tok@mf\endcsname{\def\PY@tc##1{\textcolor[rgb]{0.40,0.40,0.40}{##1}}}
\expandafter\def\csname PY@tok@mh\endcsname{\def\PY@tc##1{\textcolor[rgb]{0.40,0.40,0.40}{##1}}}
\expandafter\def\csname PY@tok@mi\endcsname{\def\PY@tc##1{\textcolor[rgb]{0.40,0.40,0.40}{##1}}}
\expandafter\def\csname PY@tok@il\endcsname{\def\PY@tc##1{\textcolor[rgb]{0.40,0.40,0.40}{##1}}}
\expandafter\def\csname PY@tok@mo\endcsname{\def\PY@tc##1{\textcolor[rgb]{0.40,0.40,0.40}{##1}}}
\expandafter\def\csname PY@tok@ch\endcsname{\let\PY@it=\textit\def\PY@tc##1{\textcolor[rgb]{0.25,0.50,0.50}{##1}}}
\expandafter\def\csname PY@tok@cm\endcsname{\let\PY@it=\textit\def\PY@tc##1{\textcolor[rgb]{0.25,0.50,0.50}{##1}}}
\expandafter\def\csname PY@tok@cpf\endcsname{\let\PY@it=\textit\def\PY@tc##1{\textcolor[rgb]{0.25,0.50,0.50}{##1}}}
\expandafter\def\csname PY@tok@c1\endcsname{\let\PY@it=\textit\def\PY@tc##1{\textcolor[rgb]{0.25,0.50,0.50}{##1}}}
\expandafter\def\csname PY@tok@cs\endcsname{\let\PY@it=\textit\def\PY@tc##1{\textcolor[rgb]{0.25,0.50,0.50}{##1}}}

\def\PYZbs{\char`\\}
\def\PYZus{\char`\_}
\def\PYZob{\char`\{}
\def\PYZcb{\char`\}}
\def\PYZca{\char`\^}
\def\PYZam{\char`\&}
\def\PYZlt{\char`\<}
\def\PYZgt{\char`\>}
\def\PYZsh{\char`\#}
\def\PYZpc{\char`\%}
\def\PYZdl{\char`\$}
\def\PYZhy{\char`\-}
\def\PYZsq{\char`\'}
\def\PYZdq{\char`\"}
\def\PYZti{\char`\~}
% for compatibility with earlier versions
\def\PYZat{@}
\def\PYZlb{[}
\def\PYZrb{]}
\makeatother


    % Exact colors from NB
    \definecolor{incolor}{rgb}{0.0, 0.0, 0.5}
    \definecolor{outcolor}{rgb}{0.545, 0.0, 0.0}



    
    % Prevent overflowing lines due to hard-to-break entities
    \sloppy 
    % Setup hyperref package
    \hypersetup{
      breaklinks=true,  % so long urls are correctly broken across lines
      colorlinks=true,
      urlcolor=urlcolor,
      linkcolor=linkcolor,
      citecolor=citecolor,
      }
    % Slightly bigger margins than the latex defaults
    
    \geometry{verbose,tmargin=1in,bmargin=1in,lmargin=1in,rmargin=1in}
    
    

    \begin{document}
    
    
    \maketitle
    
    

    
    \emph{Author: Konstantinos Oikonomou}

Copyright (C) 2018. All Rights Reserved.

    \section{SentSem}\label{sentsem}

\textbf{Measurement of the Semantic Similarity Between Sentences}

SentSem is a python script that takes any two English sentences as input
and outputs the percentage of semantic similarity they have between
them.

The libraries used for the project are: - NLTK - numpy

The Natural Language Toolkit for Python is used due to the fact that it
provides access to almost every tool needed for the process. The main
component needed is the \emph{WordNet} corpus because of its structure
and the similarity methods it provides.

    \subsection{Imports}\label{imports}

    \begin{Verbatim}[commandchars=\\\{\}]
{\color{incolor}In [{\color{incolor}1}]:} \PY{k+kn}{import} \PY{n+nn}{nltk}
        
        \PY{k+kn}{from} \PY{n+nn}{nltk}\PY{n+nn}{.}\PY{n+nn}{corpus} \PY{k}{import} \PY{n}{wordnet} \PY{k}{as} \PY{n}{wn}
        \PY{k+kn}{from} \PY{n+nn}{nltk}\PY{n+nn}{.}\PY{n+nn}{corpus} \PY{k}{import} \PY{n}{stopwords}
        
        \PY{k+kn}{from} \PY{n+nn}{nltk}\PY{n+nn}{.}\PY{n+nn}{tokenize} \PY{k}{import} \PY{n}{RegexpTokenizer}
        \PY{k+kn}{from} \PY{n+nn}{nltk}\PY{n+nn}{.}\PY{n+nn}{stem} \PY{k}{import} \PY{n}{WordNetLemmatizer}
        \PY{k+kn}{from} \PY{n+nn}{nltk}\PY{n+nn}{.}\PY{n+nn}{wsd} \PY{k}{import} \PY{n}{lesk}
        
        \PY{k+kn}{import} \PY{n+nn}{numpy} \PY{k}{as} \PY{n+nn}{np}
\end{Verbatim}


    Set up of the tokenizer, so that it keeps words and ignores any
punctuation.

Initialization of the Lemmatizer.

Initialization of the \texttt{stopwords} list.

    \begin{Verbatim}[commandchars=\\\{\}]
{\color{incolor}In [{\color{incolor}2}]:} \PY{n}{tokenizer} \PY{o}{=} \PY{n}{RegexpTokenizer}\PY{p}{(}\PY{l+s+sa}{r}\PY{l+s+s1}{\PYZsq{}}\PY{l+s+s1}{\PYZbs{}}\PY{l+s+s1}{w+}\PY{l+s+s1}{\PYZsq{}}\PY{p}{)}
        \PY{n}{lemmatizer} \PY{o}{=} \PY{n}{WordNetLemmatizer}\PY{p}{(}\PY{p}{)}
        
        \PY{n}{stopwords} \PY{o}{=} \PY{n}{stopwords}\PY{o}{.}\PY{n}{words}\PY{p}{(}\PY{l+s+s1}{\PYZsq{}}\PY{l+s+s1}{english}\PY{l+s+s1}{\PYZsq{}}\PY{p}{)}
\end{Verbatim}


    \subsection{The POS function}\label{the-pos-function}

This function takes the results from the nltk.pos\_tag() function and
turns them into pos types that can be readable by WordNet. WordNet
supports 5 different types, specifically Nouns, Verbs, Adverbs, Head
Adjectives and Satelite Adjectives (Head and Satelite are too much about
linguistics, we don't care about it very much). Here, we do not take
into accouont the Satelite Adjectives but we will do so later.

    \begin{Verbatim}[commandchars=\\\{\}]
{\color{incolor}In [{\color{incolor}3}]:} \PY{k}{def} \PY{n+nf}{pos}\PY{p}{(}\PY{n}{tag}\PY{p}{)}\PY{p}{:}
        
            \PY{k}{if} \PY{n}{tag}\PY{o}{.}\PY{n}{startswith}\PY{p}{(}\PY{l+s+s1}{\PYZsq{}}\PY{l+s+s1}{N}\PY{l+s+s1}{\PYZsq{}}\PY{p}{)} \PY{o+ow}{or} \PY{n}{tag} \PY{o}{==} \PY{l+s+s1}{\PYZsq{}}\PY{l+s+s1}{MD}\PY{l+s+s1}{\PYZsq{}}\PY{p}{:}
                \PY{k}{return} \PY{n}{wn}\PY{o}{.}\PY{n}{NOUN}
            \PY{k}{elif} \PY{n}{tag}\PY{o}{.}\PY{n}{startswith}\PY{p}{(}\PY{l+s+s1}{\PYZsq{}}\PY{l+s+s1}{J}\PY{l+s+s1}{\PYZsq{}}\PY{p}{)}\PY{p}{:}
                \PY{k}{return} \PY{n}{wn}\PY{o}{.}\PY{n}{ADJ}
            \PY{k}{elif} \PY{n}{tag}\PY{o}{.}\PY{n}{startswith}\PY{p}{(}\PY{l+s+s1}{\PYZsq{}}\PY{l+s+s1}{V}\PY{l+s+s1}{\PYZsq{}}\PY{p}{)}\PY{p}{:}
                \PY{k}{return} \PY{n}{wn}\PY{o}{.}\PY{n}{VERB}
            \PY{k}{elif} \PY{n}{tag}\PY{o}{.}\PY{n}{startswith}\PY{p}{(}\PY{l+s+s1}{\PYZsq{}}\PY{l+s+s1}{RB}\PY{l+s+s1}{\PYZsq{}}\PY{p}{)}\PY{p}{:}
                \PY{k}{return} \PY{n}{wn}\PY{o}{.}\PY{n}{ADV}
            \PY{k}{else}\PY{p}{:}
                \PY{k}{return} \PY{l+s+s1}{\PYZsq{}}\PY{l+s+s1}{\PYZsq{}}
\end{Verbatim}


    \subsection{LowerCase the Sentences}\label{lowercase-the-sentences}

    \begin{Verbatim}[commandchars=\\\{\}]
{\color{incolor}In [{\color{incolor}4}]:} \PY{n}{sent1} \PY{o}{=} \PY{n+nb}{input}\PY{p}{(}\PY{l+s+s1}{\PYZsq{}}\PY{l+s+s1}{First Sentence:}\PY{l+s+se}{\PYZbs{}n}\PY{l+s+s1}{\PYZsq{}}\PY{p}{)}\PY{o}{.}\PY{n}{lower}\PY{p}{(}\PY{p}{)}
        \PY{n}{sent2} \PY{o}{=} \PY{n+nb}{input}\PY{p}{(}\PY{l+s+s1}{\PYZsq{}}\PY{l+s+s1}{Second Sentence:}\PY{l+s+se}{\PYZbs{}n}\PY{l+s+s1}{\PYZsq{}}\PY{p}{)}\PY{o}{.}\PY{n}{lower}\PY{p}{(}\PY{p}{)}
\end{Verbatim}


    \begin{Verbatim}[commandchars=\\\{\}]
First Sentence:
I love playing soccer!
Second Sentence:
I will play football.

    \end{Verbatim}

    \subsection{Tokenization}\label{tokenization}

We use the \texttt{RegexpTokenizer} we initiallized before.

    \begin{Verbatim}[commandchars=\\\{\}]
{\color{incolor}In [{\color{incolor}5}]:} \PY{n}{tokens1} \PY{o}{=} \PY{p}{[}\PY{n}{word} \PY{k}{for} \PY{n}{word} \PY{o+ow}{in} \PY{n}{tokenizer}\PY{o}{.}\PY{n}{tokenize}\PY{p}{(}\PY{n}{sent1}\PY{p}{)}\PY{p}{]}
        \PY{n}{tokens2} \PY{o}{=} \PY{p}{[}\PY{n}{word} \PY{k}{for} \PY{n}{word} \PY{o+ow}{in} \PY{n}{tokenizer}\PY{o}{.}\PY{n}{tokenize}\PY{p}{(}\PY{n}{sent2}\PY{p}{)}\PY{p}{]}
        
        \PY{n+nb}{print}\PY{p}{(}\PY{n}{tokens1}\PY{p}{)}
        \PY{n+nb}{print}\PY{p}{(}\PY{n}{tokens2}\PY{p}{)}
\end{Verbatim}


    \begin{Verbatim}[commandchars=\\\{\}]
['i', 'love', 'playing', 'soccer']
['i', 'will', 'play', 'football']

    \end{Verbatim}

    \subsection{StopWord Removal}\label{stopword-removal}

    \begin{Verbatim}[commandchars=\\\{\}]
{\color{incolor}In [{\color{incolor}6}]:} \PY{n}{tokens1} \PY{o}{=} \PY{p}{[}\PY{n}{word} \PY{k}{for} \PY{n}{word} \PY{o+ow}{in} \PY{n}{tokens1} \PY{k}{if} \PY{n}{word} \PY{o+ow}{not} \PY{o+ow}{in} \PY{n}{stopwords}\PY{p}{]}
        \PY{n}{tokens2} \PY{o}{=} \PY{p}{[}\PY{n}{word} \PY{k}{for} \PY{n}{word} \PY{o+ow}{in} \PY{n}{tokens2} \PY{k}{if} \PY{n}{word} \PY{o+ow}{not} \PY{o+ow}{in} \PY{n}{stopwords}\PY{p}{]}
        
        \PY{n+nb}{print}\PY{p}{(}\PY{n}{tokens1}\PY{p}{)}
        \PY{n+nb}{print}\PY{p}{(}\PY{n}{tokens2}\PY{p}{)}
\end{Verbatim}


    \begin{Verbatim}[commandchars=\\\{\}]
['love', 'playing', 'soccer']
['play', 'football']

    \end{Verbatim}

    \subsection{Part Of Speech Tagging}\label{part-of-speech-tagging}

\texttt{nltk.pos\_tag()} is used. Alternatively, other POS taggers could
be more accurate in the classification of the parts of speech. It can be
seen that this one is quite inaccurate, but it is used for example
purposes only.

    \begin{Verbatim}[commandchars=\\\{\}]
{\color{incolor}In [{\color{incolor}7}]:} \PY{n}{text1} \PY{o}{=} \PY{p}{[}\PY{p}{(}\PY{n}{w}\PY{p}{,} \PY{n}{pos}\PY{p}{(}\PY{n}{p}\PY{p}{)}\PY{p}{)} \PY{k}{for} \PY{p}{(}\PY{n}{w}\PY{p}{,} \PY{n}{p}\PY{p}{)} \PY{o+ow}{in} \PY{n}{nltk}\PY{o}{.}\PY{n}{pos\PYZus{}tag}\PY{p}{(}\PY{n}{tokens1}\PY{p}{)}\PY{p}{]}
        \PY{n}{text2} \PY{o}{=} \PY{p}{[}\PY{p}{(}\PY{n}{w}\PY{p}{,} \PY{n}{pos}\PY{p}{(}\PY{n}{p}\PY{p}{)}\PY{p}{)} \PY{k}{for} \PY{p}{(}\PY{n}{w}\PY{p}{,} \PY{n}{p}\PY{p}{)} \PY{o+ow}{in} \PY{n}{nltk}\PY{o}{.}\PY{n}{pos\PYZus{}tag}\PY{p}{(}\PY{n}{tokens2}\PY{p}{)}\PY{p}{]}
        
        \PY{n+nb}{print}\PY{p}{(}\PY{n}{text1}\PY{p}{)}
        \PY{n+nb}{print}\PY{p}{(}\PY{n}{text2}\PY{p}{)}
\end{Verbatim}


    \begin{Verbatim}[commandchars=\\\{\}]
[('love', 'v'), ('playing', 'n'), ('soccer', 'n')]
[('play', 'n'), ('football', 'n')]

    \end{Verbatim}

    \subsection{Lemmatization}\label{lemmatization}

Removal of suffixes.

    \begin{Verbatim}[commandchars=\\\{\}]
{\color{incolor}In [{\color{incolor}8}]:} \PY{n}{text1} \PY{o}{=} \PY{p}{[}\PY{p}{(}\PY{n}{lemmatizer}\PY{o}{.}\PY{n}{lemmatize}\PY{p}{(}\PY{n}{word}\PY{p}{)}\PY{p}{,} \PY{n}{p}\PY{p}{)} \PY{k}{for} \PY{p}{(}\PY{n}{word}\PY{p}{,} \PY{n}{p}\PY{p}{)} \PY{o+ow}{in} \PY{n}{text1}\PY{p}{]}
        \PY{n}{text2} \PY{o}{=} \PY{p}{[}\PY{p}{(}\PY{n}{lemmatizer}\PY{o}{.}\PY{n}{lemmatize}\PY{p}{(}\PY{n}{word}\PY{p}{)}\PY{p}{,} \PY{n}{p}\PY{p}{)} \PY{k}{for} \PY{p}{(}\PY{n}{word}\PY{p}{,} \PY{n}{p}\PY{p}{)} \PY{o+ow}{in} \PY{n}{text2}\PY{p}{]}
        
        \PY{n+nb}{print}\PY{p}{(}\PY{n}{text1}\PY{p}{)}
        \PY{n+nb}{print}\PY{p}{(}\PY{n}{text2}\PY{p}{)}
\end{Verbatim}


    \begin{Verbatim}[commandchars=\\\{\}]
[('love', 'v'), ('playing', 'n'), ('soccer', 'n')]
[('play', 'n'), ('football', 'n')]

    \end{Verbatim}

    \subsection{Word Sense Disambiguation
(WSD)}\label{word-sense-disambiguation-wsd}

Words might be ambiguous, that is have multiple senses. For a word,
WordNet creates a synset for each sense of the word. In order to find
the correct synset (the correct word sense in our sentence), a common
algorithm that is used is the Lesk algorithm (developed by Micheal
Lesk).

Here we also see that if the \texttt{lesk} algorithm returns None, there
must be a problem with the POS tag that we have assigned through the
\texttt{pos()} function. This specifically means that an
\emph{adjective} has been classified incorrectly as a head adjective,
whereas it is a \textbf{satelite} adjective. That is why we use the
if-statement for.

    \begin{Verbatim}[commandchars=\\\{\}]
{\color{incolor}In [{\color{incolor}9}]:} \PY{n}{sensed1} \PY{o}{=} \PY{p}{[}\PY{p}{(}\PY{n}{word}\PY{p}{,} \PY{n}{lesk}\PY{p}{(}\PY{n}{sent1}\PY{p}{,} \PY{n}{word}\PY{p}{,} \PY{n}{p}\PY{p}{)}\PY{p}{)}
                   \PY{k}{if} \PY{p}{(}\PY{n}{lesk}\PY{p}{(}\PY{n}{sent1}\PY{p}{,} \PY{n}{word}\PY{p}{,} \PY{n}{p}\PY{p}{)} \PY{o+ow}{is} \PY{o+ow}{not} \PY{k+kc}{None}\PY{p}{)} \PY{k}{else} \PY{p}{(}\PY{n}{word}\PY{p}{,} \PY{n}{lesk}\PY{p}{(}\PY{n}{sent1}\PY{p}{,} \PY{n}{word}\PY{p}{,} \PY{l+s+s1}{\PYZsq{}}\PY{l+s+s1}{s}\PY{l+s+s1}{\PYZsq{}}\PY{p}{)}\PY{p}{)}
                   \PY{k}{for} \PY{p}{(}\PY{n}{word}\PY{p}{,} \PY{n}{p}\PY{p}{)} \PY{o+ow}{in} \PY{n}{text1}\PY{p}{]}
        \PY{n}{sensed2} \PY{o}{=} \PY{p}{[}\PY{p}{(}\PY{n}{word}\PY{p}{,} \PY{n}{lesk}\PY{p}{(}\PY{n}{sent2}\PY{p}{,} \PY{n}{word}\PY{p}{,} \PY{n}{p}\PY{p}{)}\PY{p}{)}
                   \PY{k}{if} \PY{p}{(}\PY{n}{lesk}\PY{p}{(}\PY{n}{sent2}\PY{p}{,} \PY{n}{word}\PY{p}{,} \PY{n}{p}\PY{p}{)} \PY{o+ow}{is} \PY{o+ow}{not} \PY{k+kc}{None}\PY{p}{)} \PY{k}{else} \PY{p}{(}\PY{n}{word}\PY{p}{,} \PY{n}{lesk}\PY{p}{(}\PY{n}{sent1}\PY{p}{,} \PY{n}{word}\PY{p}{,} \PY{l+s+s1}{\PYZsq{}}\PY{l+s+s1}{s}\PY{l+s+s1}{\PYZsq{}}\PY{p}{)}\PY{p}{)}
                   \PY{k}{for} \PY{p}{(}\PY{n}{word}\PY{p}{,} \PY{n}{p}\PY{p}{)} \PY{o+ow}{in} \PY{n}{text2}\PY{p}{]}
        
        \PY{n+nb}{print}\PY{p}{(}\PY{n}{sensed1}\PY{p}{)}
        \PY{n+nb}{print}\PY{p}{(}\PY{n}{sensed2}\PY{p}{)}
\end{Verbatim}


    \begin{Verbatim}[commandchars=\\\{\}]
[('love', Synset('love.v.01')), ('playing', Synset('playing.n.02')), ('soccer', Synset('soccer.n.01'))]
[('play', Synset('shimmer.n.01')), ('football', Synset('football.n.01'))]

    \end{Verbatim}

    \begin{Verbatim}[commandchars=\\\{\}]
{\color{incolor}In [{\color{incolor}10}]:} \PY{n}{synsets1} \PY{o}{=} \PY{p}{[}\PY{n}{syns} \PY{k}{for} \PY{p}{(}\PY{n}{\PYZus{}}\PY{p}{,} \PY{n}{syns}\PY{p}{)} \PY{o+ow}{in} \PY{n}{sensed1}\PY{p}{]}
         \PY{n}{synsets2} \PY{o}{=} \PY{p}{[}\PY{n}{syns} \PY{k}{for} \PY{p}{(}\PY{n}{\PYZus{}}\PY{p}{,} \PY{n}{syns}\PY{p}{)} \PY{o+ow}{in} \PY{n}{sensed2}\PY{p}{]}
         
         \PY{n}{len1} \PY{o}{=} \PY{n+nb}{len}\PY{p}{(}\PY{n}{sensed1}\PY{p}{)}
         \PY{n}{len2} \PY{o}{=} \PY{n+nb}{len}\PY{p}{(}\PY{n}{sensed2}\PY{p}{)}
\end{Verbatim}


    \subsection{Matrix Creation}\label{matrix-creation}

We create a matrix with as many rows as the remaining tokens of sentence
1 and as many columns as the remaining tokens of sentence 2. Let this
matrix be A(i, j). Each element (i, j) of the matrix is assigned to be
the path similarity between the ith token of sentence 1 and the jth
token of sentence 2. Path similarity is a common similarity measure in
WordNet, however another similarity metric might be more appropriate
depending on the application.

    \begin{Verbatim}[commandchars=\\\{\}]
{\color{incolor}In [{\color{incolor}11}]:} \PY{k+kn}{from} \PY{n+nn}{pprint} \PY{k}{import} \PY{n}{pprint}
         
         \PY{n}{sim\PYZus{}matrix} \PY{o}{=} \PY{n}{np}\PY{o}{.}\PY{n}{zeros}\PY{p}{(}\PY{p}{(}\PY{n}{len1}\PY{p}{,} \PY{n}{len2}\PY{p}{)}\PY{p}{,} \PY{n}{np}\PY{o}{.}\PY{n}{float32}\PY{p}{)}
         \PY{k}{for} \PY{n}{row} \PY{o+ow}{in} \PY{n+nb}{range}\PY{p}{(}\PY{n}{len1}\PY{p}{)}\PY{p}{:}
             \PY{k}{for} \PY{n}{col} \PY{o+ow}{in} \PY{n+nb}{range}\PY{p}{(}\PY{n}{len2}\PY{p}{)}\PY{p}{:}
                 
                 \PY{n}{sim} \PY{o}{=} \PY{n}{synsets1}\PY{p}{[}\PY{n}{row}\PY{p}{]}\PY{o}{.}\PY{n}{path\PYZus{}similarity}\PY{p}{(}\PY{n}{synsets2}\PY{p}{[}\PY{n}{col}\PY{p}{]}\PY{p}{)}
         
                 \PY{k}{if} \PY{n}{sim} \PY{o+ow}{is} \PY{o+ow}{not} \PY{k+kc}{None}\PY{p}{:}
                     \PY{n}{sim\PYZus{}matrix}\PY{p}{[}\PY{n}{row}\PY{p}{]}\PY{p}{[}\PY{n}{col}\PY{p}{]} \PY{o}{=} \PY{n}{synsets1}\PY{p}{[}\PY{n}{row}\PY{p}{]}\PY{o}{.}\PY{n}{path\PYZus{}similarity}\PY{p}{(}\PY{n}{synsets2}\PY{p}{[}\PY{n}{col}\PY{p}{]}\PY{p}{)}
                 \PY{k}{else}\PY{p}{:}
                     \PY{n}{sim\PYZus{}matrix}\PY{p}{[}\PY{n}{row}\PY{p}{]}\PY{p}{[}\PY{n}{col}\PY{p}{]} \PY{o}{=} \PY{l+m+mi}{0}
         
         \PY{n}{pprint}\PY{p}{(}\PY{n}{sim\PYZus{}matrix}\PY{p}{)}
\end{Verbatim}


    \begin{Verbatim}[commandchars=\\\{\}]
array([[ 0.11111111,  0.08333334],
       [ 0.14285715,  0.125     ],
       [ 0.09090909,  0.5       ]], dtype=float32)

    \end{Verbatim}

    \subsection{Matching Pairs of Words}\label{matching-pairs-of-words}

Each word has a best pairing match, i.e. the word with which it presents
maximum similarity.

\subsection{Final Computation}\label{final-computation}

We add up the similarities of the pairs, multiply them by two and divide
the result with the sum of the lengths of the two token lists. The
reason to do this can be found in the following link:
https://www.sciencedirect.com/science/article/pii/S1570866708000658 .
This is an academic paper created by Fredriksson and Grabowski. It has
some Graph Theory mathematics and robust supporting theory that explains
the formula.

    \begin{Verbatim}[commandchars=\\\{\}]
{\color{incolor}In [{\color{incolor}12}]:} \PY{n}{sim\PYZus{}sum} \PY{o}{=} \PY{n+nb}{max}\PY{p}{(}\PY{p}{[}\PY{n+nb}{sum}\PY{p}{(}\PY{n}{sim\PYZus{}matrix}\PY{o}{.}\PY{n}{max}\PY{p}{(}\PY{n}{axis}\PY{o}{=}\PY{l+m+mi}{0}\PY{p}{)}\PY{p}{)}\PY{p}{,} \PY{n+nb}{sum}\PY{p}{(}\PY{n}{sim\PYZus{}matrix}\PY{o}{.}\PY{n}{max}\PY{p}{(}\PY{n}{axis}\PY{o}{=}\PY{l+m+mi}{1}\PY{p}{)}\PY{p}{)}\PY{p}{]}\PY{p}{)}
         \PY{n+nb}{print}\PY{p}{(}\PY{p}{(}\PY{l+m+mi}{2}\PY{o}{*}\PY{n}{sim\PYZus{}sum}\PY{p}{)}\PY{o}{/}\PY{p}{(}\PY{n}{len1}\PY{o}{+}\PY{n}{len2}\PY{p}{)}\PY{p}{)}
\end{Verbatim}


    \begin{Verbatim}[commandchars=\\\{\}]
0.301587304473

    \end{Verbatim}


    % Add a bibliography block to the postdoc
    
    
    
    \end{document}
